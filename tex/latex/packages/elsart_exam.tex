\documentclass{elsart}
%%%\usepackage{natbib}
\begin{document}
\runauthor{Cicero, Caesar and Vergil}
\begin{frontmatter}
\title{In Catilinam IV: A murder
% A murder A murder A murder A murder A murder A murder A murder A murder A
%murder 
in 5 acts\thanksref{X}}
\author[Paestum]{Marcus Tullius Cicero\thanksref{Someone}}
\author[Rome]{Julius Caesar}
\author[Rome]{Catullus}
\author[Baiae]{Publius Maro Vergilius}
\author[Paestum]{Unknown author}
\thanks[X]{This is the history of the paper, etc etc}

\address[Paestum]{Buckingham Palace, Paestum}
\address[Baiae]{The White House, Baiae}
\address[Rome]{Senate House, Rome}
\thanks[Someone]{Partially supported by the Roman Senate}
\begin{abstract}
Cum M.~Cicero consul {\boldmath \ldots ($\vec{a}$) } Nonis Decembribus senatum in aede
Iovis Statoris consuleret, quid de iis coniurationis Catilinae
sociis fieri placeret, qui in custodiam traditi essent, factum
est, ut duae potissimum sententiae proponerentur, 
una D.~Silani consulis
designati, qui morte multandos illos censebat

Altera C.~Caesaris, qui illos publicatis bonis per municipia
Italiae distribueudos ac vinculis sempiternis tenendos existimabat. Cum
acautem plures senatores ad C.~Caesaris quam ad
D.~Silani sententiam inclinare viderentur, M.~Cicero ea, quae
infra legitur, oratione Silani sententiam commendare studuit.
\end{abstract}
\begin{keyword}
Cicero; Catiline;
Orations
\end{keyword}
\end{frontmatter}

\section{Introduction}
hello\footnote{A footnote}
\subsection{Introduction}
\subsubsection{Introduction}
\paragraph{Introduction}
\typeout{SET RUN AUTHOR to \@runauthor}

\begin{equation}\label{fo:theta}
  \theta = (a_{1},a_{2},\ldots)^{T}
\end{equation}


\section{Introduction}

Start with a section

\section{Thoughts}
Start with a list:
\begin{itemize}
\item In Catilinam I
\item In Catilinam II
\item In Catilinam III
\item In Catilinam IV
\end{itemize}

\section{Introduction}
A first section with a reference to \cite{ref1}, as well as
\cite{L1,M2,M3,M6,M9}, and to equation \ref{fo:theta}

Who is Caesar\ldots

Video, patres conscripti, in me omnium vestrum ora atque oculos esse
conversos, video vos non solunn de vestro ac rei publicae, verum
etiam, si id depulsum sit, de meo periculo esse sollicitos. Est mihi
iucunda in malis et grata in dolore vestra erga me voluntas, sed eam,
per deos inmortales, deponite atque obliti salutis meae de vobis ac de
vestris liberis cogitate. Mihi si haec condicio consulatus data est,
ut omnis acerbitates, onunis dolores cruciatusque perferrem, feram non
solum fortiter, verum etiam lubenter, dum modo meis laboribus vobis
populoque Romano dignitas salusque pariatur. \par Ego sum ille consul,
patres conscripti, cui non forum, iu quo omnis aequitas continetur,
non campus consularibus auspiciis consecratus, non curia, summum
auxilium omnium gentium, non domus, commune perfugium, non lectus ad
quietem datus, non denique haec sedes honoris [sella curulis] umquam
vacua mortis periculo atque insidiis fuit.  Ego multa tacui, multa
pertuli, multa concessi, multa meo quodam dolore in vestro timore
sanavi. Nunc si hunc exitum consulatus mei di inmortales esse
voluerunt, ut vos populumque Romanum ex caede miserrima, coniuges
liberosque vestros virginesque Vestales ex acerbissima vexatione,
templa atque delubra, hanc pulcherrimam patriam omnium nostrum ex
foedissima flamma, totam Italiam ex bello et vastitate eriperem,
quaecumque mihi uni proponetur fortuna, subeatur.\footnote{no, not
  sub-eating.} Etenim, si P.~Lentulus suum nomen inductus a vatibus
fatale ad perniciem rei publicae fore putavit, cur\footnote{Not a dog,
  but a footnote.} ego non laeter meum consulatum ad salutem populi
Romani prope fatalem extitisse?

Figure \ref{song} is not by Cicero.
\begin{figure}
\begin{quote}
\fontfamily{hlcw}\fontshape{it}\selectfont
Fear no more the heat o' the sun,\\
Nor the furious winter's rages;\\
Thou thy worldly task hast done,\\
Home art gone, and ta'en thy wages:\\
Golden lads and girls all must,\\
As chimney-sweepers, come to dust.\\
Fear no more the frown o' the great;\\
Thou art past the tyrant's stroke;\\
Care no more to clothe and eat;\\
To thee the reed is as the oak:\\
The sceptre, learning, physic, must\\
All follow this, and come to dust.\\
Fear no more the lightning flash,\\
Nor the all-dreaded thunder-stone;\\
Fear not slander, censure rash;\\
Thou hast finish'd joy and moan:\\
\\
 All lovers young, all lovers must\\
 Consign to thee, and come to dust.\\
\\
No exorciser harm thee!\\
Nor no witchcraft charm thee!\\
Ghost unlaid forbear thee!\\
Nothing ill come near thee!\\
 Quiet consummation have;\\
 And renowned be thy grave!\\
\end{quote}
\caption{A Song, by William Shakespeare from his play
  \emph{Cymbeline}, performed by two characters called
Guiderius and  Arviragus}
\label{song}
\end{figure}
\begin{table*}
\caption{Defect parameters for AgCl(111) surfaces$^{\#}$}
\label{defparagcl} 
\begin{center}
\begin{tabular}{l l l l}
\hline 
$\Delta_j H_{\mathrm{S}}^\circ$, & \multicolumn{2}{c}{Conductivity}& Profile
analysis\\  
\cline{2-4}
$\Delta_j S_{\mathrm{S}}^\circ$ & \cite{S12}& &
\cite{F4}$^\ast$\\ 
\hline 
$\Delta_{\mathrm{i}} H_{\mathrm{S}}^\circ /\mathrm{meV} $ & 370  & 800 & 780\\  
$\Delta_{\mathrm{i}} S_{\mathrm{S}}^\circ / k $ &  4.7 & 10 &  9.2\\ 
$\Delta_{\mathrm{v}} H_{\mathrm{S}}^\circ / \mathrm{meV} $ & 880 & 600 & 690 \\ 
$\Delta_{\mathrm{v}} S_{\mathrm{S}}^\circ / k$ &  2.0 &  0 & 0.6\\ 
\hline 
\end{tabular}
\end{center}

\vspace*{.6cm}
\noindent
$^{\#}$ The evaluation relies on an Ag$_{\mathrm{i}}^\cdot$
accumulation which is in agreement with the literature but which is still
under debate (see text) \cite{phdjam,jmjf}. \\
$^\ast$ A recent reevaluation resulted particularly in different entropy
values. \cite{S14}.
\end{table*}

\section{The argument}
Some words might be appropriate describing equation \ref{e1}.
\begin{equation}
\frac{\partial F}{\partial
t}=D\frac{\partial^2 F}{\partial x^2},
\label{e1}
\end{equation}

\begin{eqnarray}
x & = & 1\\
y & = & 2\\
z & = & 3
\end{eqnarray}

Quare, patres conscripti, consulite vobis, prospicite patriae,
conservate vos, coniuges, liberos fortunasque vestras, populi Romani
nomen salutemque defendite; mihi parcere ac de me cogitare desinite. Nam
primum debeo sperare omnis deos, qui huic urbi praesident, pro eo mihi,
ac mereor, relaturos esse gratiam; deinde, si quid obtigerit, aequo
animo paratoque moriar. Nam neque turpis mors forti viro potest accidere
neque immatura consulari nec misera sapienti. Nec tamen ego sum ille
ferreus, qui fratris carissimi atque amantissimi praesentis maerore non
movear horumque omnium lacrumis, a quibus me circumsessum videtis Neque
meam mentem non domum saepe revocat exanimata uxor et abiecta metu filia
et parvulus filius quem mihi videtur amplecti res publica tamquam ob
sidem consulatus mei, neque ille, qui expectans huius exitum diei stat
in conspectu meo, gener. Moveo his rebus omnibus, sed in eam partem, uti
salvi sint vobiscum omnes, etiamsi me vis aliqua oppresserit, potius,
quam et illi et nos una rei publicae peste pereamus. \par Quare, patres
conscripti, incumbite ad salutem rei publicae, circumspicite omnes
procellas, quae inpendent, nisi providetis. Non Ti.~Gracchus, quod
iterum tribunus plebis fieri voluit, non C.~Gracchus, quod agrarios
concitare conatus est, non L.~Saturninus, quod C.~Memmium occidit, in
discrimen aliquod atque in vestrae severitatis iudicium
adduciturtenentur ii, qui ad urbis incendium, ad vestram omnium caedem,
ad Catilinam accipiendum Romae restiterunt, tenentur litterae, signa,
manus, denique unius cuiusque confessio; sollicitantur Allobroges,
servitia excitantur, Catilina accersitur; id est initum consilium, ut
interfectis omnibus nemo ne ad deplorandum quidem populi Romani nomen
atque ad lamentandam tanti imperii calamitatem relinquatur. \par Haec omnia
indices detulerunt, rei confessi sunt, vos multis iam iudiciis
iudicavistis, primum quod mihi gratias egistis singu laribus verbis et
mea virtute atque diligentia perditorum hominum coniurationem patefactam
esse decrevistis, deinde quod P.~Lentulum se abdicare praetura
coegistis, tum quod eum et ceteros, de quibus iudicastis, in custodiam
dandos censuistis, maximeque quod meo nomine supplicationem decrevistis,
qui honos togato habitus ante me est nemini; postremo hesterno die
praemia legatis Allobrogum Titoque Volturcio dedistis amplissima. Quae
sunt omnia eius modi, ut ii, qui in custodiam nominatim dati sunt, sine
ulla dubitatione a vobis damnati esse videantur.

\subsection{A sub section}
Also containing
some text, and a
figure. Also containing
some text, and a
figure
Also
containing some
text, and a figure
Also containing
some text, and a
figure (\ref{f1}).

\begin{figure*}
Leave 3cm of space for non-electronic graphic on full page width:
  \vspace{3cm}
\caption{A Caption}
\label{f1}
\end{figure*}
\subsection{Another sub section}
\subsubsection{A subsub section} with some text in it.
\subsubsection{Another subsub section}with some text in it.
\paragraph{And a paragraph}
(ie a sub sub subsection)

\subsection{Another sub section}
\section{Epilogue}
A word or two to conclude,  and this even includes some
inline maths: \(R(x,t)\sim
t^{-\beta}g(x/t^\alpha)\exp(-|x|/t^\alpha)\)


 Quae cum ita sint, pro imperio, pro exercitu, pro provincia, quam
neglexi, pro triumpho ceterisque laudis insignibus, quae sunt a me
propter urbis vestraeque salutis custodiam repudiata, pro clientelis
hospitiisque provincialibus, quae tamen urbanis opibus non minore labore
tueor quam comparo, pro his igitur omnibus rebus, pro meis in vos
singularibus studiis proque hac, quam perspicitis, ad conservandam rem
publicam diligentia nihil a vobis nisi huius temporis totiusque mei
consulatus memoriam postulo; quae dunn erit in vestris fixa mentibus,
tutissimo me muro saeptum esse arbitrabor.\footnote{Cicero wrote very
  long sentences$\ldots$} Quodsi meam spem vis
inproborum fefellerit atque superaverit, commendo vobis parvum meum
filium, cui profecto satis erit praesidii non solum ad salutem, verum
etiam ad dignitatem, si eius, qui haec omnia suo solius periculo
conservarit, illum filium esse memineritis. \par Quapropter de summa salute
vestra populique Romani, de vestris coniugibus ac liberis, de aris ac
focis, de fanis atque templis de totius urbis tectis ac sedibus, de
imperio ac libertate, de salute Italiae, de universa re publica
decernite diligenter, ut instituistis, ac fortiter. Habetis eum
consulem, qui et parere vestris decretis non dubitet et ea, quae
statueritis, quoad vivet, defendere et per se ipsum praestare possit.

now for a vec, as in $\vec{66}$

\begin{thebibliography}{999}
\bibitem{L1} C.C. Liang, {\em J. Electrochem. Soc.} {\bf 120} (1973) 1298.

\bibitem{W1'} J.B. Wagner, {\em High Conductivity Conductors Solid Ionic
Conductors\/}, T. Takahashi, ed., World Scientific, Singapore, 1989; A. Shukla,
N. Vaidehi, K.T. Jacob, {\em Proc. Indian Acad. Sci.\/} {\bf 96} (1986) 533.

\bibitem{M1} a) J. Maier, {\em Ber. Bunsenges. Phys. Chem.\/} {\bf 88} (1984)
  1057;  
b) {\em phys. stat. sol. B\/} {\bf 123} (1984) K89, K187; 
c) {\em Mater. Sci. Monogr.\/} {\bf 28A} (1985) 419; 
d) {\em Structure Relations in Fast Ion and Mixed Conductors\/}, F. W. Poulsen
et al., eds. Ris\o{} Nat. Lab., 1986, Roskilde p. 153;
e) {\em Solid State Ionics} {\bf 32/33} (1989) 727; 
f) {\em Solid State Ionics\/} {\bf 18/19} (1986) 1141.   

\bibitem{M2} J. Maier, {\em J. Electrochem. Soc.\/} {\bf 134} (1987) 1524;
{\em Solid State Ionics: Materials and Applications\/}, S. Chandra, B.V.R.
Chowdari, eds., World Scientific, Singapore, 1992, p. 111.  

\bibitem{dudn} N.J Dudney, {\em Ann. Rev. Mat. Sci.\/} {\bf 19} (1989) 103.

\bibitem{M3} J. Maier, {\em J. Phys. Chem. Solids\/}  {\bf 46} (1985) 309. 

\bibitem{M6} J. Maier, {\em Ber. Bunsenges. Phys. Chem.\/}  {\bf 89} (1985)
  355.  

\bibitem{M9}  a) J. Maier, {\em Mater. Res. Bull.\/} {\bf 20} (1985) 383; b)
J. Maier, B. Reichert, {\em Ber. Bunsenges. Phys. Chem.\/} {\bf 90} (1986)
666.   

\bibitem{M11} J. Maier, {\em Ber. Bunsenges. Phys. Chem.\/} {\bf 90} (1986)
  26.  

\bibitem{M13} a) J. Maier, {\em Solid State Ionics\/} {\bf 23} (1987) 59;
b){\em phys. stat. sol. (a)\/} {\bf 112} (1989) 115.

\bibitem{M14} J. Maier, {\em Ber. Bunsenges. Phys. Chem.\/} {\bf 93} (1989)
  1468; {\em Ber. Bunsenges. Phys. Chem.\/} {\bf 93} (1989) 1474;  
{\em Solid State Ionics\/} {\bf 28-30} (1988) 1073.

\bibitem{M15} J. Maier, S. Prill, B. Reichert, {\em Solid State Ionics\/} {\bf 
28-30} (1988) 1465.

\bibitem{M16} J. Maier, {\em Mater. Chem. Phys.\/} {\bf 17} (1987) 485; {\em
Superionic Solid and Solid Electrolytes, Recent Trends\/}, A.L. Laskar, S.
Chandra, eds., Academic Press, New York, 1989, p. 137; in {\em Science and
  Technology of Fast Ion Conductors\/}, H.L. Tuller, Ed. Plenum Press, New
York 1989, p. 345. 

\bibitem{G1} J. Maier, {\em Angew. Chem. Int. Ed. Engl.\/} {\bf 32}(3)
  (1993) 313; {\em Angew. Chem. Int. Ed. Engl.\/}, {\bf 32}(4) (1993) 528; in 
{\em Defects in Insulating Materials\/}, O. Kanert and J.-M. Spaeth, eds.,
World Scientific, Singapore, 1992, p. 2. 

\bibitem{W7} J. Wassermann, T.P. Martin, J. Maier, {\em Solid State Ionics\/}
{\bf 28-30} (1988) 1514.

\bibitem{L3} U. Lauer, J. Maier, {\em Ber. Bunsenges. Phys. Chem.\/} {\bf 96},
(1992) 111; {\em Solid State Ionics\/} {\bf 51} (1992) 209.

\bibitem{M27} J. Maier, U. Lauer, {\em Ber. Bunsenges. Phys. Chem.\/} {\bf
  94} (1990) 973; U. Lauer, J. Maier, W. G\"opel, {\em Sensors and Actuators
  B} {\bf 2} (1990) 125; {\em Solid State Ionics\/} {\bf 40/41} (1990) 463. 

\bibitem{M26} P. Murugaraj and J. Maier, {\em Solid State Ionics\/} {\bf
  32/33} (1989) 993; {\em Solid State Ionics\/} {\bf 40/41} (1990) 1017.

\bibitem{M25'} U. Lauer and J. Maier, {\em J. Electrochem. Soc.\/} {\bf
  139}(5) (1992) 1472.  

\bibitem{A3} A. Atkinson, R.I. Taylor, {\em Phil. Mag.} A43 979; {\em Solid
    State Ionics\/} {\bf 28-30} (1988) 1377. 

\bibitem{A1} T. Asai, S. Kawai, {\em Solid State Ionics\/} {\bf 20} (1986) 225. 

\bibitem{A2} T. Asai, C.-H. Hu, S. Kawai, {\em Mater. Res. Bull.\/}  {\bf 22},
(1987) 269. 

\bibitem{C1} L. Chen , Z. Zhao, G. Wang, Z. Li, {\em Kexue Tongbao\/} {\bf
  26} (1981) 308.

\bibitem{C2} M.R.-W. Chang, K. Shahi, J.B. Wagner, {\em J. Electrochem.
Soc.\/} {\bf 131} (1984) 1213.

\bibitem{C3} P. Chowdhary, V.B. Tare, J.B. Wagner, {\em J. Electrochem.
Soc.\/} {\bf 132} (1985) 123.

\bibitem{C4} P. Chowdhary, J.B. Wagner, {\em Mater. Lett.\/} {\bf 3} (1985) 78. 

\bibitem{C5} P. Chowdhary, A. Khandkar, J.B. Wagner, The Electrochemical
Society Meeting, October 1985, Las Vegas.

\bibitem{D1} R. Dupree, J.R. Howells, A. Hooper, F.W. Poulsen, {\em Solid
State Ionics\/} {\bf 3/4} (1981) 277.

\bibitem{D2} P. Dubec, J.B. Wagner, {\em Mater. Lett.\/} {\bf 2} (1984) 302.

\bibitem{H1} A. Hooper, {\em J. Power Sources\/} {\bf 9} (1983) 161.

\bibitem{J1} T.R. Jow, C.C. Liang, {\em Solid State Ionics\/} {\bf 9/10}
  (1983) 695.  

\bibitem{J2} T. Jow, J.B. Wagner, {\em J. Electrochem. Soc.\/} {\bf 126}(1979)
  1963.  

\bibitem{K1} A. Khandkar, J.B. Wagner, The Electrochem. Soc. Meeting,
San Francisco, 1983; A. Khandkar, Thesis, 1985, Arizona State University.

\bibitem{L2} C.C. Liang, A.V. Joshi, N.E. Hamilton, {\em J. Applied
Electrochem.\/} {\bf 8} (1978) 445.

\bibitem{N1} O. Nakamura, J.B. Goodenough, {\em Solid State Ionics\/} {\bf 7} 
(1982) 125.

\bibitem{P1} S. Pack, B. Owens, J.B. Wagner, {\em J. Electrochem. Soc.\/} {\bf 
127} (1980) 2177.

\bibitem{P2} F.W. Poulsen, N.H. Andersen, B. Kindl, J. Schoonman, {\em Solid
State Ionics\/} {\bf 9/10} (1983) 131.

\bibitem{P3} F.W. Poulsen, {\em Solid State Ionics\/} {\bf 2} (1981) 53.

\bibitem{P4} F.W. Poulsen and P.J. M\o ller in: loc. cit. [3d], p. 159.

\bibitem{P5} F.W. Poulsen in: loc. cit. [3d], p. 67.

\bibitem{P6} J.B. Phipps, D.H. Whitmore, {\em Solid State Ionics\/} {\bf 9/10}
 (1983) 123.

\bibitem{P7} J.B. Phipps, D.H. Whitmore, {\em J. Power Sources\/} {\bf 9}
  (1983) 373.  

\bibitem{S1} P.M. Skarstadt, D.B. Merritt, B.B. Owens, {\em Solid State
Ionics\/} {\bf 314} (1981) 277.

\bibitem{S2} K. Shahi, J.B. Wagner, {\em J. Phys. Chem. Solids\/} {\bf 43},
 (1982) 713.

\bibitem{S3} K. Shahi, J.B. Wagner, {\em J. Solid State Chem.\/} {\bf 42}  
(1982) 107. 

\bibitem{S4} K. Shahi, J.B. Wagner, {\em J. Electrochem. Soc.\/} {\bf 128}
(1981) 6.

\bibitem{W1} J.B. Wagner, {\em Mater. Res. Bull.\/} {\bf 15} (1980) 1691.

\bibitem{Z1} Z. Zhao, C. Wang, L. Chen, {\em Wuli Xuebao\/} {\bf 33}
(1984) 1205; Z. Zhao, C. Wang, S. Dai, L. Chen, {\em Solid State Ionics\/} {\bf
  9/10} (1983) 1175.

\bibitem{F1} S. Fujitsu, M. Miyayama, K. Koumoto, H. Janagida, T. Kanazawa,
{\em J. Mater. Sci.\/} {\bf 20} (1985) 2103.

\bibitem{F2} S. Fujitsu, K. Kuomoto, H. Janagida, {\em Solid State Ionics\/}
{\bf 18/19} (1986) 1146.

\bibitem{K2} A. Khandkar, V.B. Tare, J.B. Wagner, {\em Rev. Chim. Min.\/} {\bf
23} (1986) 274.

\bibitem{V1} N. Vaidehi, R. Akila, A.K. Shukla, K.T. Jacob, {\em Mater. Res.
Bull.\/} {\bf 21} (1986) 909.

\bibitem{S1'} E. Hartmann, V.V. Peller, G.I. Rogalski, {\em Solid State
    Ionics\/} {\bf 28-30} (1988) 1098, 
V. Trnovcov\'a, C. B\'arta, P.P. Fedorov, I. Zibrov, {\em Materials Science
  Forum\/} {\bf 76} (1991) 13, 
S. Adams, K. Hariharan, J. Maier, {\em Solid State Ionics\/}, {bf 75} (1991)
193, Y. He, Z. Chen, Z. Zhang, L. Wang and L. Chen in {\em Materials for Solid 
  State Batteries\/}, B.V.R. Chowdhari and S. Radhakrishna, eds., World
Scientific, Singapore, 1986, p. 333.,  
J. R. Stevens and B.E. Mellander, {\em Solid State Ionics\/} {\bf 21}
(1986) 203, 
S. Skaarup, K. West and B. Zachau Christiansen, {\em Solid State
  Ionics\/} {\bf 28-30} (1989) 975, 
J. Plocharski, W. Wieczorek, J. Przyluski and K. Such, {\em J. Appl. Phys. A\/}
{\bf 49} (1989) 55,  
W. Wieczorek, K. Such, H. Wycislik and J. Plocharski, {\em Solid State Ionics\/}
{\bf 36} (1989) 255, 
J. Plocharski and W. Wieczorek, {\em Solid State Ionics\/} {\bf 28-30} 979;
(1988) \\ 
Y. Saito, T. Asai, K. Ado and O. Nakamura, {\em Mat. Res. Bull.\/} {\bf 23}
(1988) 1661, 
Y. Saito, T. Asai, O. Nakamura and Y. Yamamoto, {\em Solid State Ionics\/} {\bf
  35} (1989) 241,  
Y. Saito, J. Mayne, K. Ado, Y. Yamamoto and O. Nakamura, {\em Solid State
  Ionics\/} {\bf 40/41} (1990) 72,  
J. Mayne, Y. Saito, H. Kageyama, K. Ado, T. Asai and O. Nakamura, {\em Solid
  State Ionics\/} {\bf 40/41} (1990) 67, 
Y. Saito, K. Ado, T. Asai, H. Kageyama, O. Nakamura and Y. Yamamoto, {\em Solid
  State Ionics\/} {\bf 53-56} (1992) 728,  
O. Nakamura and Y. Saito, {\em Solid State Ionics: Materials and
  Applications\/}, S. Chandra, S. Singh and P.C. Srivastava, eds., World
Scientific, Singapore, 1992, p. 101;\\  
B. Wnetrzewski, J.L. Nowinski and W. Jakubowski, {\em Solid State Ionics\/}
{\bf 36} (1989) 209, 
P. Weslowski, W. Jakubowski and J.L. Nowinski, {\em phys. stat. sol. (a)\/}
{\bf 115} (1989) 81,  
J.L. Nowinski, P. Kurek and W. Jakubowski, {\em Solid State Ionics\/} {\bf 36} 
(1989) 213;  
P. Kurek, J.L. Nowinski and W. Jakubowski, {\em Solid State Ionics\/} {\bf 36} 
(1989) 243;  
H. Husono and Y. Abe {\em Solid State Ionics\/}, {\bf 44} (1990) 293;
B. Kumar, J.D. Schaffer, M. Nookola, L. G. Scanlou, {\em J. Power Sources\/}
{\bf 47} (1994) 63;\\
S.N. Reddy, A.S. Chary, K. Saibabu, T. Chiranjiui, A. Brune, J.B. Wagner,
{\em Solid State Ionics\/} {\bf 25} (1987) 165,  
K. Singh, S.S. Bhoga {\em Solid State Ionics\/} {\bf 39} (1990) 205; 
S.S. Bhoga, K. Singh {\em Solid State Ionics\/} {\bf 40/41} (1990) 27,  
S. Chaklanobis, K. Shahi, R.K. Syal {\em Solid State Ionics\/} {\bf 44} (1990)
107,  
L. Chen, C. Gros, R. Castaynet, P. Hagenmuller {\em Solid State Ionics\/} {\bf
  31} (1988) 209.   

\bibitem{S11} G. Simkovich, C. Wagner, {\em J. Catalysis\/} {\bf 1} (1962) 521. 

\bibitem{W2} T.L. Wen, R.A. Huggins, A. Rabenau, W. Weppner, {\em Rev. Chim.
Min.\/} {\bf 20} (1983) 40.

\bibitem{saito} Y. Saito, J. Maier, in preparation; Y. Saito. K.
Hariharan, J. Maier, {\em Proc. 4th Int. Symp. on Syst. with Fast
  Ionic Transport\/}, Warsaw, 1994, in press.

\bibitem{hari} K. Hariharan, J. Maier, {\em J. Electrochem. Soc.\/},
  submitted.  

\bibitem{S6} A.M. Stoneham, E. Walde, J.A. Kilner, {\em Mater. Res. Bull.\/} 
{\bf 14} (1979) 1661.

\bibitem{S5} K. Shahi, J.B. Wagner, {\em Appl. Phys. Lett.\/} {\bf 37}
(1980) 757. 

\bibitem{D4} N.J. Dudney, {\em J. Am. Ceram. Soc.\/} {\bf 70} (1987) 65.

\bibitem{W3} C. Wagner, {\em J. Phys. Chem. Solids\/} {\bf 33} (1972) 1051.

\bibitem{B1} A. Bunde, W. Dieterich, E. Roman, {\em Solid State Ionics\/} {\bf 
18/19} (1986) 147.

\bibitem{R1} H.E. Roman, A. Bunde, W. Dieterich: loc. cit. [3d], p. 165.

\bibitem{R2} H.E. Roman, A. Bunde, W. Dieterich, {\em Phys. Rev.\/} {\bf B34}
(1986) 331.

\bibitem{W4} J.C. Wang, N.J. Dudney, {\em Solid State Ionics\/} {\bf 18/19}
(1986) 112.

\bibitem{S7} H. Schmalzried, {\em Solid State Reactions\/}. Verlag Chemie,
Weinheim, 1981.

\bibitem{F3} J. Frenkel, {\em Kinetic Theory of Liquids\/}. Oxford University
Press, New York, 1946.

\bibitem{K3} K.L. Kliewer, J.S. K\"ohler, {\em Phys. Rev.\/} {\bf A 140} 
(1965) 1226; C. Newey, P. Pratt, A. Lidiard, {\em Phil. Mag.\/} {\bf 3} (1958)
75.  

\bibitem{P9} R.B. Poeppel, J.M. Blakely, {\em Surf. Sci.\/} {\bf 15} (1969)
  507.

\bibitem{jm} J. Maier, in {\em Science and Technology of Fast Ion
  Conductors\/}. H.L. Tuller, M. Balkanski, Eds. Plenum Press, New York, 1989,
p. 299.  

\bibitem{jamma} J. Jamnik, J. Maier and S. Pejovnik {\em Solid State Ionics},
  in press. 

\bibitem{macdon} J.R. Macdonald, D.R. Franceschetti and A.P. Lehnen, {\em
    J. Chem. Phys.\/} {\bf 73} (1980) 5272; J.R. Macdonald and A.P. Lehnen,
  {\em Cryst. Latt. Def.\/} {\bf 9} (1982) 149.

\bibitem{M19} J. Maier, W. G\"opel, {\em J. Solid State Chem.\/} {\bf 72} 
(1988) 293; W. G\"opel, J. Maier, K. Schierbaum, and H.-D. Wiemh\"ofer, {\em
  Solid State Ionics\/} {\bf 32/33} (1989) 440.

\bibitem{M28} J. Maier, U. Lauer, {\em Solid State Ionics\/} {\bf 51} (1992)
  209. 

\bibitem{J3} J. Jamnik, J. Maier and S. Pejovnik, {\em Solid State Ionics\/}
  {\bf 75} (1995) 51; {\em Electrochimica Acta\/}, {\bf 14} (1993) 1975;
  J. Jamnik, J. Maier and S. Pejovnik, {\em Proc. 1st Int. Slovenian-German 
Seminar of Joints Projects in Mat. Sci. and Technol.\/}, Ljubljana, October
94, to be published; J. Fleig and J. Maier, {\em Proc. 1st
  Int. Slovenian-German Seminar of Joints Projects in Mat. Sci. and
  Technol.\/}, Ljubljana, October 94, to be published.  

\bibitem{B2} H. B\"ottger, V.V. Bryksin, {\em Hopping Conduction in Solids\/}.
VCH, Weinheim, 1985.  

\bibitem{B3} R. Blender, W. Dieterich, {\em Solid State Ionics\/}, {\bf 28-30} 
(1988) 82.

\bibitem{kleitz} M. Kleitz, H. Bernard, E. Fernandez, E. Schrouter, {\em Adv.
  Ceram. Sci. Tech. Zirconia\/} {\bf 3} (1981) 310; H. Rickert, U. Schwaitzer,
{\em Solid State Ionics\/} {\bf 9/10} (1983) 689; H. J. Queisser, J.H. Werner,
{\em Mat. Res. Soc. Symp. Proc.\/} (1988) 53; H. J. M"oller, H. P. Strunk,
J. Werner eds., {\em Polycrystalline Semiconductors\/}, Springer, Berlin,
1989. 

\bibitem{phdfleig} J. Fleig, Ph D-Thesis, T\"ubingen-Stuttgart, 1995 (detailed
  calculation of such effects by finite elements); J. Fleig, J. Maier, in
  preparation. 

\bibitem{soltis} R.E. Soltis, E.M. Logothetis, A.D. Brailsford, J.B.
Wagner, {\em J. Electrochem. Soc.\/} {\bf 135} (1988) 2380; A.D. Brailsford, 
{\em Solid State Ionics} {\bf 21} (1986) 159; N.F. Uvarov, E.F. Hairetdinov,
J.M. Reau, P. Hagenmuller, {\em Solid State Comm.\/} {\bf 79} (1991) 635. 

\bibitem{U1} N.F. Uvarov and J. Maier, {\em Solid State Ionics\/}, {\bf 62}
  (1993) 251. 

\bibitem{U1'} N.F. Uvarov, V.P. Isopov, V. Sharma, A.K. Shukla {\em Solid
State Ionics\/} {\bf 51} (1992) 41. 

\bibitem{P10} G.E. Pike, private communication.

\bibitem{C6} Y.M. Chiang, A.F. Henriksen, W.D. Kingery, D. Finello, {\em J.
Am. Ceram. Soc.\/} {\bf 64} (1981) 385.

\bibitem{S9} P. Sutton, private communication.

\bibitem{K4} K.L. Kliewer, {\em J. Phys. Chem. Solids\/} {\bf 27} (1966) 705. 

\bibitem{F4} G. Farlow, A.B. Blose, Sr.J. Feldott, B.D. Lounsberry, L.
Slifkin, {\em Radiation Eff.\/} {\bf 75} (1983) 1; Sr. J. Feldott, A.B.
Blose, B.D. Lounsberry, L. Slifkin, {\em J. Imag. Sci.} {\bf 29} (1985) 39;
R.A. Hudson, G.C. Farlow, L.M. Slifkin, manuscript. 

\bibitem{R4} U. Riedel, J. Maier, and R. Brook, {\em J. Eur. Ceram. Soc.\/}
{\bf 9}(3) (1992) 205; {\em Solid State Ionics}, M. Balkanski, T. Takahashi,
H.L. Tuller, Eds. Elsevier Science Publishers, Amsterdam, 1992 p. 259.

\bibitem{T2} P.W. Tasker {\em Advances in Ceramics\/} {\bf 10} (1984) 176. 

\bibitem{B4} L. Bousse, N.F. de Rooij, P. Bergveld, {\em Trans. Electron.
Dev.\/} {\bf 30} (1983) 1263.

\bibitem{V3} A. van der Berg, P. Bergveld, D.N. Reinhondt, E.J.B.
Sudh\"alter, {\em Sens. Actuat.\/} {\bf 8} (1985) 129.

\bibitem{D3} E.A. Daniels, S.M. Rao, {\em Z. Phys. Chem.\/} {\bf N. F. 137}
  (1983) 247. 

\bibitem{R3} S.M. Rao, {\em Ph. D. Thesis\/}, 1983, Poona University.

\bibitem{W5} J.L. White in: {\em Proc. 4th Nat. Conf. Clays and Clay Min.,
Int. Ser. Monogr. Earth Sciences\/}, p. 133.

\bibitem{Z2} C. Zipelli, J.C.J. Bart, C. Petrini, S. Galvagno, C. Cimino,
{\em Z. Anorg. allg. Chem.\/} {\bf 502} (1983) 199.

\bibitem{M21} J. Maier, G. Schwitzgebel, {\em Mater. Res. Bull.\/} {\bf 17} 
(1982) 1061; J. Maier, {\em Solid State Ionics\/} {\bf 32/33} (1989) 727. 

\bibitem{M22} J. Mizusaki, K. Fueki, {\em Solid State Ionics\/} {\bf 6}
(1982) 55. 

\bibitem{S8} O. Stasiw, J. Teltow, {\em Ann. Phys.\/} (Lpz.) {\bf 1}
(1947) 261. 

\bibitem{M23} D. M\"uller, {\em phys. stat. sol.\/} {\bf 1312} (1965) 775.

\bibitem{M20} J. Maier, {\em Z. Phys. Chem.\/} {\bf N.F. 140} (1984) 191;
{\em J. Am. Ceram. Soc.\/}, {\bf 76}(5) (1993) 1212; {\em J. Am. Ceram.
  Soc.\/} {\bf 76}(5) (1993) 1218; {\em J. Am. Ceram. Soc.\/}, {\bf 76}(5)
(1993) 1223; {\em J. Am. Ceram. Soc.\/}, {\bf 76}(5) (1993) 1228.  

\bibitem{V2} N. Valverde-Diez, J.B. Wagner, Jr., {\em Solid State
  Ionics\/}, {\bf 28-30} (1988) 1697.

\bibitem{sens} J. Maier, {\em Solid State Ionics\/}, {\bf 62} (1993) 105.

\bibitem{U2} N.F. Uvarov, M.C.R. Shastry and K.J. Rao {\em Rev. Solid
State Sci.\/} {\bf 4} (1990) 61.

\bibitem{schmidt} J.A. Schmidt, J.C. Baz\'an, L. Vico, {\em Solid State
  Ionics\/} {\bf 27} (1988) 1.

\bibitem{H2} Y. Haven, {\em Rec. Tran. Chem.\/} {\bf 69} (1950) 1471, 1505.

\bibitem{mog} M. Mogensen, {\em J. Power Sources\/} {\bf 20} (1987) 53;
  M. Gaber$\check{s}\check{c}$ek, J. Jamnik, S. Pejovnik, {\em
    J. Electrochem. Soc.\/} {\bf 140} (1993) 308.

\bibitem{zhu} B. Zhu, B.E. Mellander, in: C. Singhal and H. Iwahara, eds. {\em
    Solid Oxide Fuel Cells\/} (The Electrochemical Soc. Inc., Pennington, NJ,
  1993) 156. 

\bibitem{unruh} T. Unruh, Ph D-Thesis, Saarbr\"ucken, FRG (1995)

\bibitem{gian} E. Gianellis, {\em Solid State Ionics\/}, in press. 

\bibitem{H6} R.A. Hubermann, {\em Phys. Rev. Lett.\/} {\bf 32} (1974) 1000.

\bibitem{H5} N. Hainovsky, J, Maier, {\em Solid State Ionics\/}, in press;
  {\em Phys. Rev.}, in press. 

\bibitem{H7} R. Lipowski, {\em Phys. Rev. Lett.\/} {\bf 49} (1982) 1575; V.N.
Bondarev and A.B. Kuklov, {\em Solid State Ionics\/} {\bf 44} (1991) 145. 

\bibitem{E2} G. Ertl, J. K\"uppers, {\em Low Energy Electrons and Surface
Chemistry\/}. VCH, Weinheim, 1985. 

\bibitem{K5} W.D. Kingery, {\em Mater. Sci. Monogr.\/} {\bf 28A} (1985) 25.

\bibitem{G2} J.-E. Gerner, Diploma Thesis, 1984, University of Konstanz.

\bibitem{strange} J.H. Strange, S.M. Rageb, R.C.T. Slade, {\em Phil. Mag.\/}
{\bf A 64} (1991) 1159.

\bibitem{H4} R.A. Hudson, G.C. Farlow and L.M. Slifkin {\em Phys. Rew. B\/}
{\bf 36} (1987) 4651. 

\bibitem{S14} L. Slifkin, personal communication; S.K. Wonnell and L.M.
Slifkin {\em Phys. Rev. B\/} {\bf 48} (1993) 78.

\bibitem{jf} J. Fleig, J. Jamnik, J. Maier, in preparation. 

\bibitem{haberm} J. Jamnik, H.-U. Habermeier, J. Maier, {\em Physica B\/} {\bf
    204} (1995) 57.   

\bibitem{K7} M. Kleitz, {\em Solid State Ionics\/} {\bf 3/4} (1981) 513.

\bibitem{G5} S. Gupka, S. Patnaik, K. Shahi, {\em Solid State Ionics\/} {\bf
    31} (1) (1988) 5. 

\bibitem{ul} U. Lauer, Ph D thesis, University of T\"ubingen, 1991. 

\bibitem{G6} F. Granzer, {\em J. Imag. Sci.\/} {\bf 33} (1989) 207.

%\bibitem{esh} Eshelby quoted in \cite{K3}. 

\bibitem{O1} J.T. Overbeek in: {\em Colloid Science I\/}, H. R. Kruyt, Ed. 
Elsevier, Amsterdam, 1952.

\bibitem{B6} R.C. Baetzold, J.F. Hamilton, {\em Surf. Sci.\/} {\bf 33} (1972)
  461.  

\bibitem{B7} R.C. Baetzold, {\em J. Phys. Chem. Solids\/} {\bf 35} (1974) 89. 

\bibitem{H3} H.A. Hoyen, {\em J. Appl. Phys.} {\bf 47\/} (1976) 3784.

\bibitem{S12} N. Starbov, {\em J. Inf. Rec. Mater.\/} {\bf 13} (1985) 307.

\bibitem{T1} Y.T. Yan, H.A. Hoyen, {\em Surf. Sci.\/} {\bf 36} (1973) 242.

\bibitem{M24} S. M\"uhlherr, K. L\"auger, N. Nicoloso, E. Schreck, K.
Dransfeld, {\em Solid State Ionics\/} {\bf 28-30} (1988) 1495.

\bibitem{phdjam} J. Jamnik, Ph D-Thesis, Ljubljana-Stuttgart, 1994.

\bibitem{jmjf} J. Fleig and J. Maier, in preparation.

\bibitem{G4} W. G\"opel, U. Lampe, {\em Phys. Rev.\/} {\bf B 22} (1980) 6447. 

\bibitem{S13} E. Schreck, K. L\"auger, K. Dransfeld, {\em Z. Phys.\/} {\bf
  B62} (1986) 331. 

\bibitem{P8} J.B. Phipps, D.L. Johnson, D.H. Whitmore, {\em Solid State
Ionics\/} {\bf 5} (1981) 393.  

\bibitem{C7} J. Corish, P.W.M. Jacobs, {\em Surf. Def. Prop. Solids\/} {\bf
  2} (1973) 160.

\bibitem{ref1}  W. Puin, P. Heitjans, {\em Proc. Int. Conf. Nanocryst. Mater.},
  Stuttgart, 1994, in press; W. Puin, P. Heitjans, J. Maier, in preparation.

\bibitem{C8} S. Chandra, private communication. 

\bibitem{G3} W. G\"opel, {\em Progr. Surf. Sci.\/} {\bf 20} (1986) 1.

\bibitem{}  M. v. Smoluchowski, Z. Phys. Chem. {\bf 92}, 129 (1917).

\end{thebibliography}
\appendix
\section{APPENDIX}
\begin{figure}
\caption{see equation \ref{foo}}
\end{figure}
\begin{equation}
a + b = c
  \label{foo}
\end{equation}
\end{document}
\subtitle{Romano dignitas salusque pariatur.
Ego sum ille consul, patres
conscripti, cui non forum, iu quo omnis aequitas continetur, non campus
consularibus auspiciis consecratus, non curia, summum auxilium omnium
gentium, non domus, commune perfugium, non lectus ad quietem datus, non
denique haec sedes honoris [sella curulis] umquam vacua mortis periculo
atque insidiis fuit.  Ego multa tacui, multa pertuli, multa concessi.
}
