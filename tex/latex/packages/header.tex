%%%%%%%%%%%%%%%%%%%%%%%%%%%%%%%%%%%%%%%%%%%%%%%%%%%%%%%%%%%%%%%
% Header file to tailor output of ABC2MTEX music scores.
%
% ABC2MTEX is a preprocessor to MusiXTeX which converts
% music scores written in the ABC language into MusiXTeX source.
% 
% The ABC2MTEX package is freely available from the LaTeX
% distribution site(s).
%
% Usage :
% -------
% 1. Create a file (e.g. song.abc) containing the ABC music score.
% 2. Issue the command "abc2mtex -x song:1".
%    This will produce a file called music.tex containing the
%    corresponding MusiXTeX source for the first entry in "song".
% 3. Process the file music.tex in the usual way to obtain
%    your music score in the LaTeX typeset output.
%    The best result is obtained with the following 3-stage
%    processing, which directly provides a pdf output file :
%    a) pdflatex music 
%    b) musixflx music
%    c) pdflatex music
%
% For further details about the usage of the ABC music package
% or the ABC2MTEX converter, please refer to the manual
%
% "ABC2MTEX - An easy way to transcribe folk and traditional music"
%  by : Chris Walshaw (C.Walshaw@gre.ac.uk)
%
% Note : The interpretation of some of the information fields
%        has been re-defined in this header file.
%
% The new conventions used here are :
% -----------------------------------
%
% X:Reference number (obligatory)
% T:Overall title of a music score (obligatory; should only be given once)
% S:Additional info w.r.t. the title or source (optional)
% C:Composer info (optional)
% A:Instrument specification (optional)
% P:Part/Movement specification (optional)
% N:Additional info w.r.t. the part/movement (optional)
%
% Example of an abc input file structure :
% ---------------------------------------- 
% X:1
% T:Impressions from the Roman Empire
% S:(Extracts from various symphonies)
% C:H. Hornblower (arr.)
% A:Trumpet I
% P:I. Intrada at the Colosseum
% N:(Tempo di Marcia)
% Q:1/8=60
% M:2/4
% K:C
% (here follows the music score in ABC notation)
%
% X:2
% T:
% P:II. An evening with Cleopatra
% N:(Sehr langsam)
% M:4/4
% K:C
% (here follows the music score in ABC notation)
%
% Additional newly defined signs, accents etc...
% ----------------------------------------------
%
% Large scale structure indications
% O : Round Coda   (e.g. |O aGeF|)
% Q : Square Coda  (e.g. |Q aGeF|)
% S : Normal dal segno (e.g. |S aGeF|)
% Y : Large dal segno (e.g. |Y aGeF|)
% X : Repeat previous bar/Due volte (e.g. |X|)
% Z : Rest of 10 bars (e.g. |Z|)
% J : Rest of  5 bars (e.g. |J|)
% N : New from the start/Da Capo (e.g. |aGeF N|])
% 
% Accents on notes
% H : Sforzato (hat ^ above/below a note e.g. Ha)
% P : Portato (- above/below a note e.g. Pa)
% S : Sforzando (> above/below a note e.g. Sa)
% L : Fermata (long) above the staff (e.g. |aGeF z3 L|)
% L : Fermata (long) above a note (e.g. |aGeF Lc4|)
% T : Trill above the staff (e.g. |T aGeF|)
% T : Trill above a note (e.g. |aGTeF|)
%
% Indications of dynamics
% R : Ritenuto (e.g. |aG R eF|)
% H : Hurry/Accelerando (e.g. |aG H eF|)
% I : Increase volume/Crescendo (e.g. |I aGeF| or |aGIeF|)
% W : Weaken volume/Diminuendo (e.g. |W aGeF| or |aGWeF|)
% U : Breath/Caesura (e.g. |aG U eF|)
% V : Volume (e.g. |aG MPV eF|)
%     where the various volume levels are combinations of :
%     P : Piano    K : Kraftig/Forte   M : Mezzo
%
% Author : Nick van Eijndhoven, NCFS 01-nov-2007.
%%%%%%%%%%%%%%%%%%%%%%%%%%%%%%%%%%%%%%%%%%%%%%%%%%%%%%%%%%%%%%%
\documentclass[a4paper]{nveart}
\usepackage{musixtex}
\pagestyle{plain}

\input musixlit
\input musixeng

\mathversion{bold}

\begin{document}

% Page set-up
% -----------
\vsize=0.8\paperheight
\hsize=0.9\paperwidth
\hoffset=-1in                  % Remove blank area such that sidemargin is counted from paper edge
\voffset=-1in                  % Remove blank area such that topmargin is counted from paper edge
\textwidth=\hsize              % Text width
\textheight=\vsize             % Text height
\oddsidemargin=0.05\paperwidth % Left margin on odd-numbered pages.
\evensidemargin=\oddsidemargin % Left margin on even-numbered pages.
\marginparwidth=0pt            % Don't reserve space for margin notes
\marginparsep=0pt
\marginparpush=0pt
\topmargin=0.08\paperheight    % Offset from top of page
\footskip=\topmargin           % Offset of page number from text

% Music set-up
% ------------
%\smallmusicsize                    % Use small size
\def\freqbarno{10}                  % Bar number after every 10 bars
\def\stdafterruleskip{2\Internote}  % Space between bar and next note
\parindent 0pt                      % No indentation

% Fonts set-up
% ------------
\font\Tifont=cmr10 % For Tline entries within tunes
\font\Wfont=cmr10  % For Wline entries within tunes
\font\Pfont=cmr7   % For Pline entries within tunes
\font\gfont=cmr7   % For guitar chords
\font\sevenrm=cmr7 % For numerical indications like e.g. in triplets 

% these lines prevent/allow pagebreaking in the middle of tunes
%\let\tune=\vbox
\let\tune=\empty

% Tune header set-up
% ------------------
\def\header{%
\if Y\Ttrue{\centerline{\bf{\huge \Tstring}}\vspace*{2mm}}\fi%
\if Y\Strue{\centerline{\large \Sstring}\nobreak}\fi%
\vspace*{1cm}
\if Y\Ctrue{\hbox to\hsize{\hfil{\large \Cstring}}\vspace*{5mm}}\fi%
\if Y\Atrue{\hbox to\hsize{{\large \Astring}\hfil}\vspace*{3mm}}\fi%
\if Y\Ptrue{\centerline{\bf{\Large \Pstring}}\vspace{2mm}}\fi%
\if Y\Ntrue{\centerline{\large \Nstring}\vspace{2mm}}\fi%
}

% Text within tunes
% -----------------
\def\Tline#1{\medskip\line{\Tifont #1\hfil}}
\def\Wline#1{\smallskip\line{\Wfont #1\hfil}}
\def\Pline#1{\notes\uptext{\Pfont #1}\enotes\relax}

% (Re)definition of some auto-generated commands
% to comply with the MusiXTeX syntax.
% ---------------------------------------------
%\let\startmuflex=\empty % To omit warning from second \startmuflex in \startpiece
\renewcommand{\endmuflex}{\end{document}}
\let\beginHp=\empty
\let\endHp=\empty

% Up and Down bows
% ----------------
\def\ubow#1{\zcharnote#1{\upbow}}
\def\vbow#1{\zcharnote#1{\downbow}}

% Gracing notes
% -------------
\def\grace{\notes\multnoteskip\tinyvalue\tinynotesize}
\let\egrace=\enotes

% Rolls
% -----
\def\uroll#1{\zcharnote{#1}{\raise -3.0\internote\hbox to 2.5\internote%
 {\hss$\smile$\hss}}}
\def\lroll#1{\zcharnote{#1}{\raise  1.0\internote\hbox to 2.5\internote%
 {\hss$\frown$\hss}}}

% Additional user defined signs, accents etc...
% ---------------------------------------------
%
% Shorthands for # and b in Guitar chords
\def\#{$^\sharp$} % sharp symbol #
\def\b{$^\flat$}  % flat symbol b
%
% Large scale structure indications
\def\userO{\Uptext{\coda1}}                        % Round Coda
\def\userQ{\Uptext{\Coda1}}                        % Square Coda
\def\userS{\bsk\Uptext{\segno1}\sk}                % Normal dal segno
\def\userY{\notes\sk\zchar0{\Segno}\sk\enotes}     % Large dal segno
\def\userX{\notes\hsk\zchar0{\duevolte}\sk\enotes} % Repeat previous bar (Due volte)
\def\userZ{\notes\zchar5{\Hpause0 3}\sk\uptext{\large \bf 10}\sk\sk\sk\enotes \advance\barno 9} % Rest of 10 bars
\def\userJ{\notes\zchar5{\Hpause0 2}\sk\uptext{\large \bf 5}\sk\hsk\enotes \advance\barno 4}    % Rest of 5 bars
\def\userN{\notes\zchar{-5}{\large \bf D.C.}\sk\hsk\enotes}     % Da capo (new from the start)
% 
% Accents on notes
\def\userHl#1{\usfz#1}                          % Sforzato (hat ^ above note when stem down)
\def\userHu#1{\lsfz#1}                          % Sforzato (hat v below note when stem up)
\def\userPl#1{\ust#1}                           % Portato (- above note when stem down)
\def\userPu#1{\lst#1}                           % Portato (- below note when stem up)
\def\userSl#1{\usf#1}                           % Sforzando (> above note when stem down)
\def\userSu#1{\lsf#1}                           % Sforzando (> below note when stem up)
\def\userL{\notes\uptext{\fermataup0}\enotes}   % Fermata (above the staff)
\def\userLl#1{\fermataup#1}                     % Fermata (above the note when stem down)
\def\userLu#1{\uptext{\fermataup0}}             % Fermata (above the staff when stem up)
\def\userT{\notes\uptext{\TrilleX0 2}\enotes}   % Trill (above the staff)
\def\userTl#1{\zcharnote#1{\trilleX4 1}}        % Trill (above the note when stem down)
\def\userTu#1{\uptext{\trilleX0 1}}             % Trill (above the staff when stem up)
%
% Indications of dynamics
\def\userR{\notes\cchar{-5}{rit.}\hsk\enotes}           % Ritenuto
\def\userH{\notes\cchar{-5}{accel.}\hsk\enotes}         % Accelerando (Hurry)
\def\userI{\notes\cchar{-5}{cresc.}\hsk\enotes}         % Crescendo (Increase volume)
\def\userIl#1{\zchar{-4}{\crescendo{2\noteskip}}}       % Crescendo (below the staff when stem down)
\def\userIu#1{\zchar{-4}{\crescendo{2\noteskip}}}       % Crescendo (below the staff when stem up)
\def\userW{\notes\cchar{-5}{dim.}\hsk\enotes}           % Diminuendo (Weaken volume)
\def\userWl#1{\zchar{-4}{\decrescendo{2\noteskip}}}     % Diminuendo (below the staff when stem down)
\def\userWu#1{\zchar{-4}{\decrescendo{2\noteskip}}}     % Diminuendo (below the staff when stem up)
\def\userU{\notes\cbreath\enotes}                       % Breath comma above the staff
\def\userUl#1{\cbreath}                                 % Breath comma above the staff (when stem down)
\def\userUu#1{\cbreath}                                 % Breath comma above the staff (when stem up)
\newcounter{V}\setcounter{V}{0}%
\newcommand{\userP}{\addtocounter{V}{-1}}               % Piano
\newcommand{\userK}{\addtocounter{V}{1}}                % Forte (Kraftig)
\newcommand{\userM}{\addtocounter{V}{10}}               % Mezzo
\newcommand{\userV}{%                                   % Volume
\ifthenelse{\value{V}=-1}{\notes\cchar{-4}{\p}\bsk\enotes}{}%
\ifthenelse{\value{V}=-2}{\notes\cchar{-4}{\pp}\bsk\enotes}{}%
\ifthenelse{\value{V}=-3}{\notes\cchar{-4}{\ppp}\bsk\enotes}{}%
\ifthenelse{\value{V}=-4}{\notes\cchar{-4}{\pppp}\bsk\enotes}{}%
\ifthenelse{\value{V}=1}{\notes\cchar{-4}{\f}\bsk\enotes}{}%
\ifthenelse{\value{V}=2}{\notes\cchar{-4}{\ff}\bsk\enotes}{}%
\ifthenelse{\value{V}=3}{\notes\cchar{-4}{\fff}\bsk\enotes}{}%
\ifthenelse{\value{V}=4}{\notes\cchar{-4}{\ffff}\bsk\enotes}{}%
\ifthenelse{\value{V}=9}{\notes\cchar{-4}{\mp}\bsk\enotes}{}%
\ifthenelse{\value{V}=11}{\notes\cchar{-4}{\mf}\bsk\enotes}{}%
\ifthenelse{\value{V}=0}{\notes\cchar{-4}{\fp}\bsk\enotes}{}%
\setcounter{V}{0}%
}%
